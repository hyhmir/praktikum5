\documentclass[12pt]{article}
\usepackage[slovene]{babel}
\usepackage[utf8]{inputenc}
% \usepackage[T2A]{fontenc}
\usepackage{amsmath}
\usepackage{amsfonts}
\usepackage{amssymb}
\usepackage[version=4]{mhchem}
% \usepackage{stmaryrd}
\usepackage{graphicx}
\usepackage[export]{adjustbox}
\graphicspath{ {./images/} }
\usepackage{physics}
\usepackage{geometry}
\geometry{left=2cm,right=2cm,top=2cm,bottom=2cm}
\usepackage{parskip} 
\usepackage{float}
\usepackage{siunitx}


\title{\textbf{Poskusi z žarki X}}
\author{Samo Krejan}
\date{januar 2026}

\begin{document}
\maketitle

\section{Uvod}

Elektrone iz katode pospešimo z visoko napetostjo proti kovinski anodi. Pri
trku z tarčo elektroni zaradi zaviranja v polju jeder pride do zavornega sevanja v spektru X žarkov. če imajo elektroni iz vijšjih stanj zadosti energije pa lahko iz notranjih lupin gradnikov tarče izbijejo elektrone. Elektroni iz višjih stanj gradnika nato zapolnejo vrzel. Takrat pride do izseva karakterističnih X žarkov, ki imajo točno določeno energijo.

\subsection{Ionizacijska celica}

Najenostavnejša ionizacijska celica je pravzaprav kar ploščni kondenzator zvezan z izvorom visoke napetosti. če v prostor med ploščama posvetimo z rentgenskimi žarki ti ionizirajo molekule zraka (preko fotoefekta, torej fotoelektroni so tisti, ki zares ionizirajo molekule zraka). Nastale ionske pare napetost na kondenazatorju usmeri k ploščam in dobimo tokovni sunek na tokokrogu. če je fotonov dovolj se sunki spovprečijo na merljiv tok. V splošnem vsi ionski pari ne dosežejo elektrod ker se jih nekaj že prej rekombinira. Pri nizkem polju v kondenzatorju je rekombinacija znatna, pri višjih pa je praktično ni več. To pomeni da s večanjem napetosti na kondenzatorju tok najprej narašča, nato pa nastopi nasičenje.

Ekspozicijska doza $X$ je električni naboj $\Delta Q$ enega predznaka, ki ga v zraku volumna $\Delta V$ z maso $\Delta m$, na enoto mase sprosti ionizirajoče sevanje:

$$
X = \frac{\Delta Q}{\Delta m}
$$

Hitrost ekspozicije definiramo z odvodom doze in jo izrazimo z tokom:

$$
\frac{X}{t} = \frac{\Delta I}{\Delta m} = \frac{\Delta I}{\rho \Delta V}
$$

\subsection{Polariziranost žarkov X}

Zarki X nastanejo zaradi interakcije pospešenih elektronov z katode z jedri v anodi. Ti pri upočasnjevanju elektromagnetno sevajo in se jim tako zmanjša hitrost pri gibanju mimo jedra. Frekvenca izsevanega žarka je določena s kinetično energijo, ki jo izgubi foton. Maksimalno frekvenco dobimo, ko se vsa kinetična energija spremeni v elektromagnetno.

$$
E_k = h\nu
$$

Priročna je formula z anodno napetostjo $U$.

Poenostvljeno gledano naboj niha v smeri osi $y$. Pospeševanju naboja sledi sevanje elektromagnetnega valovanja, ki ga opišemo z vektorjem jakosti električnega polja $\vec{E}$ (ki ima smer nihajočega naboja in je pravokoten na smer razˇsirjanja valovanja). Ker naboj niha v smeri $y$ je tja usmerjen $\vec{E}$. Pravimo da je valovanje linearno polarizirano. Energijski tok valovanja, ki ga seva tak nihajoč naboj je največji v ekvatorialni ravnini, v smeri nihanja naboja pa je enak 0. Če imamo več istočasno nihajočih nabojev, katerih smeri nihanja so porazdeljenev ravnini $yz$. V tem primeru dobimo nepolarizirano valovanje v smeri $x$. V smereh $y$, $z$ pa je valovanje še vedno linearno polarizirano. Če nihanje ni enakomerno porazdeljeno po ravnini, dobimo delno polarizirano svetlobo v smeri $x$. Če bi elektroni v anodi zavirali le v smeri svojega gibanja bi dobili linearno polarizirane žarke. V resnici se veliko elektronov odkloni od prvotne smeri že prej in so zato žarki le delno polarizirani.

Z merjenjem jakosti elastičnega sipanja valovanja lahko določimo polariziranost rentgenske svetlobe. Pri elasticno sipanem odbojnem valovanju dobimo močno sevanje v ravnini, ki je pravokotna na smer nihanja naboja, v sami smeri nihanja pa sevanja ni. V snop, ki ima smer osi $y$ postavimo sipalec, nato pa v ravnini $xz$ z $GM$ števcem izmerimo kotno porazdelitev sipanega valovanja. Merimo pravzaprav le dve pravokotni komponenti $I_x$, $I_z$. Polariziranost tako definiramo kot:

$$
\eta = \frac{I_z-I_x}{I_z+I_x}
$$

\subsection{Presevno slikanje predmetov}

Žarki X se absorbirajo v snovi, kar lahko opišemo s preprosto enačbo za intenziteto:

$$
I = I_0\exp(-\mu d)
$$

kjer je $\mu$ absorbcijski faktor in $d$ dolžina poti žarka v sredstvu. Ta pojav lahko izkoristimo za rentgensko slikanje.

\section{Potrebščine}

\begin{itemize}
    \item Rentgenska naprava (Lehr und Didaktik Systeme 554811 ),
    \item ionizacijska celica,
    \item izvir napetosti,
    \item upor in voltmeter,
    \item 2 sipalca,
    \item GM števec,
    \item računalnik,
    \item fotoaparat.
\end{itemize}

\section{Naloga}

\begin{itemize}
    \item Z inoziacijsko celico izmeri povprečno jakost doze v snopu žarkov X,
    \item Izmeri polariziranost primarnih žarkov X,
    \item Izmeri polariziranost sipalnih žarkov X,
    \item Slikaj čim več predmetov.
\end{itemize}

\section{Meritve}

Za napetosti na rentgenski cevi $\SI{15}{kv}$, $\SI{25}{kv}$, $\SI{30}{kv}$ in $\SI{30}{kv}$ pri toku $\SI{1}{mA}$ izmerimo odvisnosti ionizacijskega toka od napetosti na kondentzatorju $I_\mathrm{C}(U_\mathrm{C})$. Meritve so zapisane v tabeli~\ref{tab:I-by-U-data} in grafično prikazane na~\ref{fig:I-U}.

Izmerimo še, da ima ionizacijska celica (kondenzator) dimenzije

\begin{align*}
    a &= (160 \pm 0.5)\,\si{mm}, \\
    b &= \left( \frac{83+137}{2} \pm 0.5 \right)\,\si{mm}, \\
    c &= (34 \pm 0.5)\,\si{mm}.
\end{align*}

\begin{table}[H]
    \centering
    \begin{tabular}{llll}
        \begin{tabular}{r|r}
            \multicolumn{2}{c}{$U_\mathrm{R} = \SI{15}{kV}$} \\
            \hline
            $U_\mathrm{C}\,[\si{V}]$ & $I_\mathrm{C}\,[\si{nA}]$ \\
            \hline
            0.0 & 0.02 \\
            9.7 & 0.08 \\
            21.2 & 0.12 \\
            29.8 & 0.14 \\
            39.3 & 0.15 \\
            48.7 & 0.16 \\
            59.1 & 0.15 \\
            69.5 & 0.16 \\
            80.6 & 0.16 \\
            90.5 & 0.16 \\
            101.4 & 0.17 \\
            120.7 & 0.15 \\
            148.2 & 0.17 \\
            178.7 & 0.18 \\
            206.5 & 0.16 \\
            253.2 & 0.17 \\
            300.9 & 0.18 \\
            & \\
            & \\
            & \\
            & \\
            & \\
        \end{tabular}
        &
        \begin{tabular}{r|r}
            \multicolumn{2}{c}{$U_\mathrm{R} = \SI{25}{kV}$} \\
            \hline
            $U_\mathrm{C}\,[\si{V}]$ & $I_\mathrm{C}\,[\si{nA}]$ \\
            \hline
            0.0 & 0.02 \\
            10.5 & 0.03 \\
            19.5 & 0.36 \\
            21.1 & 0.63 \\
            31.1 & 0.86 \\
            39.6 & 1.05 \\
            50.7 & 1.12 \\
            59.3 & 1.17 \\
            71.8 & 1.17 \\
            80.8 & 1.18 \\
            89.8 & 1.18 \\
            100.9 & 1.19 \\
            119.7 & 1.21 \\
            148.0 & 1.21 \\
            176.5 & 1.21 \\
            202.6 & 1.22 \\
            251.7 & 1.23 \\
            300.8 & 1.24 \\
            & \\
            & \\
            & \\
            & \\
        \end{tabular}
        &
        \begin{tabular}{r|r}
            \multicolumn{2}{c}{$U_\mathrm{R} = \SI{30}{kV}$} \\
            \hline
            $U_\mathrm{C}\,[\si{V}]$ & $I_\mathrm{C}\,[\si{nA}]$ \\
            \hline
            8.1 & 0.33 \\
            19.7 & 1.11 \\
            30.7 & 1.33 \\
            36.6 & 1.48 \\
            39.8 & 1.66 \\
            45.3 & 1.78 \\
            50.2 & 1.93 \\
            56.1 & 2.01 \\
            60.4 & 2.02 \\
            66.8 & 2.11 \\
            69.6 & 2.17 \\
            74.5 & 2.18 \\
            78.5 & 2.19 \\
            83.4 & 2.2 \\
            89.1 & 2.2 \\
            100.9 & 2.22 \\
            123.6 & 2.22 \\
            149.8 & 2.23 \\
            175.0 & 2.24 \\
            200.5 & 2.24 \\
            251.7 & 2.24 \\
            300.3 & 2.28 \\
        \end{tabular}
        &
        \begin{tabular}{r|r}
            \multicolumn{2}{c}{$U_\mathrm{R} = \SI{35}{kV}$} \\
            \hline
            $U_\mathrm{C}\,[\si{V}]$ & $I_\mathrm{C}\,[\si{nA}]$ \\
            \hline
            8.1 & 0.42 \\
            19.0 & 0.99 \\
            33.0 & 1.74 \\
            41.4 & 2.13 \\
            49.3 & 2.48 \\
            60.1 & 2.87 \\
            69.8 & 3.09 \\
            79.7 & 3.24 \\
            91.3 & 3.34 \\
            100.5 & 3.45 \\
            117.9 & 3.49 \\
            150.3 & 3.51 \\
            173.7 & 3.53 \\
            201.2 & 3.54 \\
            249.1 & 3.54 \\
            299.2 & 3.59 \\
            & \\
            & \\
            & \\
            & \\
            & \\
            & \\
        \end{tabular}
    \end{tabular}
    \caption{Meritve ionizacijskega toka v odvisnosti od napetosti na kondenzatorju.}
    \label{tab:I-by-U-data}
\end{table}

Za meritev polarizacije primarnih žarkov s sipanjem izmerimo intenzitete

\begin{align}
    I_z &= 115\,\si{s^{-1}}, \\
    I_{z0} &= 0.25\,\si{s^{-1}}, \\
    I_{x} &= 133.2\,\si{s^{-1}}, \\
    I_{x0} &= 0.25\,\si{s^{-1}},
    \label{eq:polar-1}
\end{align}

pri čemer so intenzitete z indeksi 0 ozadja, ki jih kasneje od pripadajoče meritve odštejemo. Za meritev polarizacije že sipanih žarkov pa

\begin{align}
    I_z &= 0.40\,\si{s^{-1}}, \\
    I_{z0} &= 0.35\,\si{s^{-1}}, \\
    I_{y} &= 0.38\,\si{s^{-1}}, \\
    I_{y0} &= 0.28\,\si{s^{-1}}.
    \label{eq:polar-2}
\end{align}

\begin{figure}[H]
    \begin{center}
        \includegraphics{I-by-U.pdf}
    \end{center}
    \caption{Izmerjene vrednosti toka $I_\mathrm{C}$ med ploščama kondenzatorja za različne napetosti na kondenzatorju. Odvisnost je izmerjena za 4 različne napetosti na rentgenski cevi; tok na rentgenski cevi pa je konstanten, sicer $1\,\mathrm{mA}$. Črte, ki povezujejo meritve, so zgolj z a boljšo preglednost.}
    \label{fig:I-U}
\end{figure}

\begin{figure}[H]
    \begin{center}
        \includegraphics[width=0.65\textwidth]{miska.JPG}
    \end{center}
    \caption{Rentgenski posnetkek računalniške miške.}
    \label{fig:sanitizer}
\end{figure}

\begin{figure}[H]
    \begin{center}
        \includegraphics[width=0.4\textwidth]{slusalke1.JPG}
        \includegraphics[width=0.4\textwidth]{slusalke2.JPG}
    \end{center}
    \caption{Rentgenska posnetka leve in desne slušalke.}
    \label{fig:stopwatch-powerbrick}
\end{figure}

\subsection{Račun}

Za volumen ionizacijske celice iz meritev izračunamo

\begin{equation*}
    V = 600\,(1 \pm 0.02)\,\si{mm^3}.
\end{equation*}

Iz meritev na sliki~\ref{fig:I-U}, posebej na sliki~\ref{fig:I-U-saturated} izračunamo nasičene ionizacijske tokove $I_{nasicen}$ (vrednosti v tabeli~\ref{tab:I-saturated}). Hitrost doze lahko potem izračunamo z izrazom

\begin{equation*}
    \dot{X} = \frac{I_{nasicen}}{\rho V},
\end{equation*}

kjer za $\rho$ uporabimo gostoto zraka

\begin{equation*}
    \rho = \SI{1.204}{kg/m^3}.
\end{equation*}

Izračunane hitrosti doze predstavimo v tabeli~\ref{tab:I-saturated}.

\begin{table}[H]
    \centering
    \begin{tabular}{r|r r}
        $U_\mathrm{R}\,[\si{kV}]$ & $I_{nasicen}\,[\si{nA}]$ & $\dot{X}\,[\si{mAs/kg h}]$ \\
        \hline
        15 & $0.17 \pm 0.01$ & $0.83 \pm 0.07$ \\
        25 & $1.20 \pm 0.02$ & $6.0 \pm 0.2$ \\
        30 & $2.23 \pm 0.03$ & $11.1 \pm 0.4$ \\
        35 & $3.52 \pm 0.04$ & $17.6 \pm 0.6$ \\
    \end{tabular}
    \caption{Iz napetostnega intervala $U_\mathrm{C} \in [100, 300]$ izračunane vrednosti zasičenega toka $I_\mathrm{C}$. Primerjamo jih lahko s prikazom meritev na sliki~\ref{fig:I-U-saturated}.}
    \label{tab:I-saturated}
\end{table}

\begin{figure}[H]
    \begin{center}
        \includegraphics{I-by-U-saturated.pdf}
    \end{center}
    \caption{Prikaz meritev iz~\ref{fig:I-U}, iz katere lažje razberemo zasičen tok $U_\mathrm{C}$.}
    \label{fig:I-U-saturated}
\end{figure}

Iz meritev~\ref{eq:polar-1} za polarizacijo primarnih žarkov lahko polarizacijo izračunamo preko

\begin{equation*}
    \eta = \frac{I_z - I_x}{I_z + I_x}.
\end{equation*}

Za polarizacijo izračunamo vrednost

\begin{equation*}
    \eta = 0.07 \pm 0.01.
\end{equation*}

Polarizacijo že sipanih žarkov izračunamo podobno, le da tokrat gledamo ravnino $xy$ namesto ravnine $xz$. Izračunamo polarizacijo

\begin{equation*}
    \eta = 0.33 \pm 0.15.
\end{equation*}


\end{document}