\documentclass[12pt]{article}
\usepackage[slovene]{babel}
\usepackage[utf8]{inputenc}
\usepackage[T2A]{fontenc}
\usepackage{amsmath}
\usepackage{amsfonts}
\usepackage{amssymb}
\usepackage[version=4]{mhchem}
\usepackage{stmaryrd}
\usepackage{graphicx}
\usepackage[export]{adjustbox}
\graphicspath{ {./images/} }
\usepackage{physics}
\usepackage{geometry}
\geometry{left=2cm,right=2cm,top=2cm,bottom=2cm}
\usepackage{parskip}
\usepackage{float} 

\title{\textbf{Elektrooptični pojav}}
\author{Samo Krejan}
\date{november 2025}

\begin{document}
\maketitle

\section{Uvod}

Tekoče kristale tvorijo podolgovate molekule, ki se pri ne previsokih temperaturah orientacijsko uredijo. Za smektične tekoče kristale je značilno, da se molekule uredijo v plasti.

Smektični kristal je oblika mezofaze (faza med trdnim in tekočim stanjem), ki so usmerijo podolgem v plasti, vendar se molekula lahko znotraj plasti prosto giba. Primer smektičnega tekočega kristala je milo.

Poznamo več vrsti smektičnih kristalov, ki jih prikazuje spodnja slika

\begin{figure}[ht]
\begin{center}
\includegraphics[width=.9\linewidth]{smecticCrystals.jpg}
\end{center}
\caption{\small Slika prikazuje, kako različne vrste smektičnih kristalov.}
\end{figure}

V smektikih A kaže odlikovana smer, ki ji pravimo direktor vzdolž normale plasti, v smektikih C* pa je kot, ki ga oklepa direktor z normalo nekje med \(10^{\circ} \text{ in } 30^{\circ}\).

Feroelektrične smektične C* tvorijo molekule, ki imajo velik električni dipolni moment prečno na vzdolžno os molekul, zato se v teh snoveh pojavi električna polarizacija, ki leži v ravnini plasti in je pravokotna na direktor. Polarizacija je sorazmerna s kotom nagiba. Tekoči kristali so posebej uporabni zaradi dvolomnosti, ki izhaja iz orientacijske urejenosti molekul, kjer je optična os vzporedna z direktorjem.

V dovolj debelem vzorcu oriše smer nagiba in s tem tudi električna polarizacija poln krog (nekaj sto do tisoč plasti). Polarizacijo plasti lahko uredimo v isto smer bodisi z zunanjim električnim poljem bodisi z ograditvijo vzorca v ploščici, ki predpisujeta orientacijo molekul. Predpisovanje orientacij molekul je doseženo s kemično ali mehansko obdelavo površin.

Če je razmik med ploščicama dovolj majhen (reda v \(\mu m\)), se direktor postavi v predpisani smeri po vsem vzorcu. V takem površinko stabiliziranem feroelektričnem tekočem kristalu so plasti kristala pravokotne na ploščici, električna polarizacija pa leži v ravnini ploščic.

Če postavimo ta tanko površinko stabiliziran feroelektrični kristal v zunanje električno polje \(\vec{E}\), ki je pravokoten na ploščici, se električna polarizacija vzorca deloma zasuče v smeri polje. Tudi direktor se deloma zasuče na stožcu smeri, ki ga določa nagib direktorja glede na normalo plasti. Zasuk električne polarizacije je sorazmeren z električnim poljem, posledično je sorazmeren tudi zasuk optične osi.

Linearnemu odzivu lomnega količnika na zunanje električno polje pravimo \emph{elektrooptični pojav}. Zasuk polarizacije je v izmeničnem polju odvisen tudi od frekvence. Pri previsoki frekvenci polarizacija ne more več slediti polju. Odvisnost spremembe polarizacije \(\partial P\) od frekvence opišemo z Debyjevim relaksacijskim modelom

\begin{equation}
\label{eq:1}
\partial P = \partial P_0 \frac{1}{1 + i \omega \tau}
\end{equation}

kjer je \(\tau\) relaksacijski čas odvisen od viskoznosti tekočega kristala in od debeline vzorca.

Kot zasuka optične osi, ki je sorazmeren s spremembo polarizacije, ima enako frekvenčno odvisnost.

Spremembo smeri optične osi zaznamo tako, da opazujemo, kako se spremeni polarizacija svetlobe pri prehodu skozi vzorec. Na vzorec posvetimo s polarizirano svetlobo in merimo svetlobno moč, ki jo prepušča analizator za vzorcem.

Vpadno polarizacijo razstavimo na izredno komponento, ki je vzporedna z optično osjo in na redno komponento, pravokotno na optično os. Prepuščeno svetlobno moč merimo s pomočjo fotodiode. Odziv nekega sistema na majhne periodične zunanje motnje najlažje izmerimos faznim občutljivim ojačevalnikom (FOO, angl. \emph{lock-in amplifier}).

FOO deluje po principu tega, da vhodni izmenični signal iz fotodiode pomnoži z referenčnim izmeničnim signalom s frekvenco modulacije (v našem primeru zunanje električnega polja, priklopljenega na tekočekristalni vzorec).

V tekočem kristalu je zasuk optične osi \(\psi\) zaradi viskoznosti snovi zakasnjen glede na zunanje električno polje. Del, ki je v fazi dobimo kot realni del enačbe \ref{eq:1}, del zasukan za \(\frac{\pi}{2}\) pa kot imaginarni del

\begin{align}\label{al:1}
  \psi_r &= \frac{\psi_0}{1 + (\omega \tau ) ^2} \\
\psi_i &= - \frac{\psi_0 \omega \tau}{1 + (\omega \tau)^2}
\end{align}

\section{Potrebščine}

\begin{itemize}
    \item Laser,
    \item fazno občutljivi ojačevalec z multimetrom,
    \item foto dioda,
    \item analizator,
    \item vzorec,
    \item osciloskop.
\end{itemize}

\begin{figure}[ht]
\begin{center}
    \includegraphics[width=11cm]{postavitev.png}
    \caption{Shematski prikaz postavitve eksperimenta}
    % \label{}
\end{center}
\end{figure}


\section{Naloga}

\begin{itemize}
    \item Prepričaj se, da je elektrooptični odziv sorazmeren z modulacijo do neke napetosti,
    \item Nariši obe komponenti signala kot funkciji frekvence in določi relaksacijski čas,
    \item Nariši razmerje med signaloma in določi relaksacijski čas.
\end{itemize}

\section{Meritve in rezultati}

Vse meritve sem zabeležil v .csv datotekah in jih kasneje obdelav v Pythonu.

\subsection{Sorazmernost odziva z modulacijo}

Pri fiksni frekvenci $\nu = 20 Hz$ sem preveril, če sta obe komponenti signala odziva res sorazmerna. S prilagajanjem premic sem preveril, da to drži. Enačbi prilaganih premic sta 

$y1 = (9.66 \pm 0.03) x - 0.16 \pm 0.05$ in 

$y2 = (-20.8 \pm 0.5) x + 2.2 \pm 0.7$

\begin{figure}[H]
\begin{center}
    \includegraphics[width=10cm]{straight.pdf}
    \caption{Sorazmernost odziva z modulacijo}
    \label{}
\end{center}
\end{figure}


\subsection{Določanje relaksacijskega časa}


Kasneje sem pri fiksni napetosti $U = 0.242\ V$ spreminjal frekvenco in opazoval odziv. Na dobljen graf sem nato prilagajal funkciji \ref{al:1} in dobil slednji vrednosti za $\tau$:

$\tau_1 = (2.6\pm 0.1) \ ms$ in


$\tau_2 = (2.7\pm 0.5) \ ms$

\begin{figure}[H]
\begin{center}
    \includegraphics[width=10cm]{vijuga.pdf}
    \caption{Določanje relaksacijskega časa s prilagajanjem funkcije}
    % \label{}
\end{center}
\end{figure}


kjer sem napake dobil s pomočjo Pythonovih orodji za prilagajanje funkcij.

Lahko pa $\tau$ izmerim tudi tako, da zgrafiram $\psi_i/\psi_r$, saj naj bi bil graf le tega premica z naklonom $-\omega\tau$

\begin{figure}[H]
\begin{center}
    \includegraphics[width=10cm]{errs.pdf}
    \caption{Razmerje komponent signala}
    % \label{}
\end{center}
\end{figure}

Na ta zadnji način sem dobil 

$\tau = (2.9 \pm 0.2) ms$

Kar se pravzapra ravno še sklada z prej določenimi vrednostmi, tako, da lahko označim rezultat kot uspešen.

\section{Zaključek}

Uspešno sem določil relaksacijski čas kar je bil cilj naloge. Sama naloga se mi je zdela dokaj lahka za izvesti, a je vzela veliko razumevanja pri obdelavi podatkov. Na kakšnih točkah sem bil verjetno malo premalo konzervativen z napakami, saj se dobljeni rezultati skoraj da ne ujemajo. Kljub vsemu označim nalogo za uspešno opravljeno.



\end{document}