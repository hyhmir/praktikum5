\documentclass[12pt]{article}
\usepackage[slovene]{babel}
\usepackage[utf8]{inputenc}
\usepackage[T2A]{fontenc}
\usepackage{amsmath}
\usepackage{amsfonts}
\usepackage{amssymb}
\usepackage[version=4]{mhchem}
\usepackage{stmaryrd}
\usepackage{graphicx}
\usepackage[export]{adjustbox}
\graphicspath{ {./images/} }
\usepackage{physics}
\usepackage{geometry}
\geometry{left=2cm,right=2cm,top=2cm,bottom=2cm}
\usepackage{parskip} 

\title{\textbf{Osnove mikrovalovne tehnike}}
\author{Samo Krejan}
\date{november 2025}

\begin{document}
\maketitle

\section{Uvod}

Mikrovalovi so elektromagnetno valovanje z valovno dolžino nekaj cm in frekvenco nekaj GHz. Kot izvor mikrovalov služijo \textbf{klistroni}. To so elektronike, ki imajo za pospeševalno mrežico še dve mrežici, povezani s poloma resonančne votline (ki je tudi povezana na pospeševalno napetost in ob vklopu, zaradi naključnih oscilacij napetosti vzbudi začetno nihanje v votlini), ki ustvarjata med mrežicama izmenično napetost, ki enakomerni curek elektronov hitrostno modulira.  Hitrost elektronov med mrežicama se namreč bodisi poveča, če kaže električno polje med mrežicama v nasprotni smeri curka, bodisi zmanjša, če
kaže polje v smeri curka. Zaradi hitrostne modulacije nastanejo po preletu mrežic v elektronskem curku zgoščine in razredčine. V refleksnem klistronu je za mrežicama resonančne votline odbojna elektroda, ki neenakomerni elektronski curek usmeri nazaj proti mrežicama in katodi. Če je odbojna napetost pravilno izbrana, se hitrostno modulirani curek vrne med mrežici s tako fazo, da električno polje gruč elektronov ojači lastno nihanje EM polja v resonančni votlini in klistron deluje kot oscilator. Pogoj za pozitivno povratno zvezo, s katero lastno nihanje v resonančni votlini je izpolnjeno za diskretne vrednosti napetosti, torej klistron deluje v različnih rodovih.

Značilna in lahko merljiva količina za stojno valovanje v vodniku je razmerje med minimalno in maksimalno amplitudo napetosti ali toka, ki ga imenujemo \textbf{ubranost}:

$$
s = \frac{|U_{min}|}{|U_{max}|}.
$$
Reaktanco bremena normirano na karakteristično upornost lahko zapišemo kot:

$$
\frac{\eta_R}{Z_0} = \frac{(s^2-1)\tan (\beta x_{min})}{1 + s^2 \tan^2 (\beta x_{min})},
$$

Enako normirana rezistenca pa je:

$$
\frac{\xi_R}{Z_0} = \left(1-\frac{\eta_R}{Z_0}\tan (\beta x_{min})\right)s
$$


Frekvenco mikrovalov merimo z resonatorjem, ki ga ugradimo v valovod. Umerimo ga s premikanjem dna. Ko je uglašen se tudi v njem pojavi valovanje, tako da se del moči valovanja na valovodu porabi. Če je vijak umerjen v frekvenčni skali lahko hitro določimo frekvenco valovanja v valovodu.

Moč merimo najpogosteje s termoelektričnimi elementi, ki se zaradi obsevanja segrejejo in se jim zato spremeni upornost. Takim elementom pravimo \textbf{bolometri}. Če z bolometrom izmerimo moč $P_m$ dobimo celotno vpadno moč kot:

$$
P = \frac{P_m}{1-|r_R|}; \ \ \ |r_R|^2 = \frac{(s-1)^2}{(s+1)^2}
$$

Termistorji so izdelani iz polprevodnikov, ki so pomešani z bakrenim prahom za boljšo prevodnost. Zveza med močjo in spremembo upornosti ni popolnoma linearna za razliko od bareterjev, ki so iz tanke platinaste žičke. Slednji so občutljivi na preobremenitve, kar termistorji niso.

\section{Potrebščine}

\begin{itemize}
    \item Klistron,
    \item ubiralka,
    \item dušilka,
    \item resonator,
    \item merilni vod,
    \item kratkostična stena,
    \item antena,
    \item bolometer,
    \item osciloskop in voltmeter.
\end{itemize}

\section{Naloga}

\begin{itemize}
    \item Prilagodite valovod na generator mikrovalov,
    \item izmerite frekvenco valovanja s pomočjo resonatrja pri enem izmed najmočnejšh rodov,
    \item poiščite rodove klistrona v odvisnosti od odbojne napetosti,
    \item izmerite moči, ki jih porablja termistor v vrhovih najmočnejših rodov,
    \item z osciloskopom pomerite krivulji ubranosti za valovod, ki je zaključen z bremenom in za kratko sklenjeni valovod.
\end{itemize}


\section{Meritve in rezultati}

Najprej sem izmeril frekvenco pri enem izmed rodov. Izbral sem si tistega pri $144,5 V$ Pred meritvijo frekvence je bilo potrebno nastaviti še ubiralko, kar se je izkazalo za precej zahtevno nalogo, saj je bil odziv izjemno občutljiv na majhne premike. 

Ko smsemo to uredili, sem na resonatorju izmerili pozicijo $413,0 \pm 0,5$. To številko s pomočjo referenčnih vrednosti v navodilih prevedemo v frekvenco in tako dobimo prvi rezultat:

$$
\nu = (8,435\pm 0.003)\ GHz
$$

Za tem sem sicer meril moči z bolometrom pri različnih rodovih, a za namene analize moramo se moramo najprej posvetiti krivulji ubranosti, saj moramo dejansko moč nekoliko popraviti glede na tisto izmerjeno.

Ubranost merimo tako, da analiziramo stoječe valovanje pri različnih zaključkih. To storimo tako, da merimo napetost na sondi v odvisnosti od pozicije. Na ta način dobimo sledeči graf (slika \ref{fig: ubranost})

\begin{figure}[ht]
\begin{center}
    \includegraphics[width=9cm]{ubranost.pdf}
    \caption{Krivulja ubranosti za različne zaključke valovoda}
    \label{fig: ubranost}
\end{center}
\end{figure}

Graf sicer prikazuje krivuljo ubranosti s filtrom, ki naredi tekoče povprečje z oknom velikosti 10. To povprečje vzamemo kot dejansko vrednost, iz razpršenosti podatkov pa ocenimo napako. Ubranost nato izračunamo po enačbi:

$$
s = \frac{|U_{min}|}{|U_{max}|} = \sqrt{\frac{h_{min}}{h_{max}}}
$$

kjer sta $h_{min}$ in $h_{max}$ najmanjši in največji izmerek napetosti po absolutni vrednosti. Relacija velja zaradi kvadratne karakteristike diode s katero merimo napetost. Tako dobimo naslednje vrednosti ubranosti predstavljene v tabeli \ref{tab: ubranost}

\begin{table}[!ht]
\centering
\begin{tabular}{|c | c|}\hline
    zaključek & ubranost \\ \hline \hline
    bolometer & $0.17\pm 0.08$ \\ \hline
    antena & $0.77\pm 0.02$ \\ \hline
    stena & $0.07\pm 0.15$ \\\hline
\end{tabular}
\caption{vrednosti ubranosti za različne zaključke}
\label{tab: ubranost}
\end{table}

Poleg ubranosti lahko iz istega grafa razberemo tudi $\lambda '$. Ta je namreč enaka dvakratniku razdalje med minimuma krivulje, ki opisuje kratkostično steno. Na ta način dobimo:

$$
\lambda ' = (4,96\pm 0,08)\ cm
$$

Z valovno dolžino mikrovalov v valovodu lahko izračunam frekvenco valovanja, ki je povezana z valovno dolžino preko svetlobne hitrosti. Čeprav bi si seveda želel uporabiti relacijo $c = \lambda \nu$ je v valovodu situacija nekoliko drugačna in velja relacija:

$$
\nu = \frac{c \sqrt{\lambda^2 + 4a^2}}{2a\lambda} = (8,35\pm 0,17)\ GHz
$$
Kar se sklada s frekvenco izmerjeno z resonatorjem v okviru napake. V formuli je $c$ hitrost svetlobe v vakumu, $a = (2,6\pm 0,1) cm$ pa je večja širina preseka valovoda. 

Poleg ubranosti je pomemben parameter, ki je bistven pri določanju impedance valovoda, razlika med lego minimuma bremena in kratkostične stene $x'_{min}$, tega za anteno in bolometer določimo iz krivulje ubranosti in dobimo $(2,1\pm 0,1)\ cm$ za anteno in $(1,18\pm 0,08)\ cm$ za bolometer. Napake sem, tako kot pri določanju valovne dolžine, ocenil iz širine minimuma. Nato sem iz s pomočjo enačb v uvodu izračunal reaktanco in rezistenco normirani na karakteristično upornost, kjer sem uporabil še sledeče dejstvo:

$$
\beta x_{min} = 2\pi \frac{x'_{min}}{\lambda'}
$$
S tem sem dobil vrednosti predstavljene v tabeli \ref{tab: impedanca}. Impedanca je seveda pitagorejska vsota reaktance in rezistence:

\begin{table}[!ht]
\centering
\begin{tabular}{c|ccc}
    & reaktanca & rezistenca & impedanca \\ \hline
    antena & $0,19\pm 0,05$ & $0,86\pm 0,05$ & $0,88\pm 0,06$\\
    bolometer & $-2,3\pm 2,7$ & $4,9\pm 2,8$ & $5,4\pm 2,6$
\end{tabular}
\caption{Reaktanca, rezistenca in impedanca normirani na karakteristično upornost}
\label{tab: impedanca}
\end{table}

Kar še preostane, je poiskati rodove klistrona in izmeriti njihovo moč. Ko sem izmeril moč, s tem pravzaprav nisem meril dejanske moči, a le to lahko dobim preko zveze:

$$
P = \frac{P_m}{1-|r_R|}; \ \ \ |r_R|^2 = \frac{(s-1)^2}{(s+1)^2}
$$
kjer je $P_m$ z bolometrom izmerjena moč. Upoštevajoč to, sem dobil naslednji graf moči v odvisnosti od napetosti v posameznih rodovih \ref{fig: moč}:
\newpage
\begin{figure}[ht]
\begin{center}
    \includegraphics[width=10cm]{moc.pdf}
    \caption{Moči v posameznih rodovih}
    \label{fig: moč}
\end{center}
\end{figure}

\section{Zaključek}

Vidimo, da so napake zares velike, a to je za pričakovati ko si ogledamo, kakšne napake smo dobili pri računu ubranosti. Tam sem bil verjetno nekoliko konzervativen pri oceni napake, saj je res nisem želel podceniti, A glede na razmazanost podatkov, se mi zdi, da je bil moj pristop pravilen. Razmazanost meritev je ponazorjena na sliki \ref{fig: razmazanost}

\begin{figure}[ht]
\begin{center}
    \includegraphics[width=9cm]{razmazanost.pdf}
    \caption{Krivulja ubranosti z neobdelanimi podatki}
    \label{fig: razmazanost}
\end{center}
\end{figure}

Kljub temu, sam dobil precej dobre rezultate, vsaj kolikor sem jih primerjal z ostalimi, ki so to vajo že opravili (tudi v prejšnjih letih) se moji rezultati večinoma precej vredu ujemajo.


\end{document}