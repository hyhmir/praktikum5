% !TEX program = xelatex

% Base
\documentclass{article}
\usepackage[a4paper,margin=1in]{geometry}

% Locale
\usepackage{polyglossia}
\setdefaultlanguage[localalph=true]{slovenian}
\usepackage[autostyle]{csquotes}
\DeclareQuoteAlias{german}{slovene}

% Bibliography
\usepackage[backend=biber,style=numeric]{biblatex}
\addbibresource{/home/martin/literatura.bib}

% Math
\usepackage{amsmath}
\usepackage{amssymb}
\usepackage{siunitx}

% Imported pdf_tex figures
\usepackage{graphicx,import}
\usepackage{subfig}
\usepackage{color}
\usepackage{float}

% Hyperlinks
\usepackage{hyperref}
\usepackage[svgnames]{xcolor}

% Pgfplots
\usepackage{amssymb}

% Styling
\numberwithin{equation}{section}
\setlength{\skip\footins}{1.5cm}

% Differential
\newcommand{\diff}{\mathrm{d}}

% "Defined as" symbol
\usepackage{mathtools}
\newcommand{\das}{\vcentcolon=}
\newcommand{\asd}{=\vcentcolon}

\title{Sunkovna jedrska magnetna resonanca}

\author{Martin Šifrar}

\begin{document}

\maketitle

\section{Naloga}

\begin{enumerate}
    \item Za vzorec vode s primešanimi paramagnetnimi ioni poišči signal proste precesije po sunku $\pi/2$ in signal spinskega odmeva po zaporedju sunkov $\pi/2$ in $\pi$. Z opazovanjem širine signala proste precesije in signala spinskega odmeva poišči takšno lego sonde, da bo magnetno polje v področju vzorca čimbolj homogeno. Iz obeh širin izračunaj $T_2^∗$ in oceni nehomogenost magnetnega polja v vzorcu.
    \item Z opazovanjem odvisnosti signala proste precesije med dvema sunkoma $\pi/2$ določi relaksacijski čas $T_1$ za vzorec vode s primešanimi paramagnetnimi ioni in za vzorec vodovodne vode.
    \item Za vodo s primešanimi paramagnetnimi ioni poišči odvisnost višine signala spinskega odmeva od presledka $\tau$ med sunkoma $\pi/2$ in $\pi$ in določi spinsko-spinski relaksacijski čas $T_2$.
\end{enumerate}

\section{Meritve}

\subsection{Enojni \texorpdfstring{$\pi/2$}{pi/2} sunek}

Prvo izvedemo z vzorcem vode z ioni meritev z enim sunkom $\pi/2$. Pogledamo rep sunka, ta pada s časovno konstanto $T_2^*$~(slika~\ref{fig:T2-star-ions}). Tako s prilagajanjem premice dobimo, da je

\begin{equation*}
    T_2^* = (0.13 \pm 0.01)\,\mathrm{ms}.
\end{equation*}

To časovno konstanto lahko ocenimo (poudarek na ocenimo) tudi iz širine spinskega odmeva v kasnejši meritvi. Ta ocena se ujema s tisto, ki smo jo dobili s prilagajanjem premice

\begin{equation*}
    T_2^*(\text{kot $1\sigma$ spinskega odmeva}) = (0.10 \pm 0.03)\,\mathrm{ms}.
\end{equation*}

\begin{figure}
    \begin{center}
    \includegraphics{T2-star-ions}
    \end{center}
    \caption{Logaritmirana napetost, sorazmerna magnetizaciji $M$ v $xy$ ravnini. To je t. i. \blockquote{signal proste precesije}. Naklon prilagojene  premice je $T_2^*$, časovna konstanta razpada magnetizacije v $xy$ ravnini (kasneje izračunamo tudi $T_2$, ki je prav tako časovna konstanta razpada $xy$ magnetizacije, le zaradi procesa, ki ga proces z razpadom $T_2^*$ prekrije).}
    \label{fig:T2-star-ions}
\end{figure}

\subsection{Dvojni \texorpdfstring{$\pi/2$}{pi/2} sunek}

Razpad magnetizacije v $xy$ ravnini je posledica dveh mehanizmov. Prvi je desinhronizacije precesije jeder zaradi nehomogenosti magnetnega polja, ki narekuje različne Larmourjeve frekvence za jedra na različnih mestih. Karakterizira ga razpadni čas $T_2^*$, ki smo ga pomerili, za katerega v grobi oceni velja

\begin{equation*}
    T_2^* = \frac{1}{\gamma \langle \delta B \rangle}.
\end{equation*}

Iz tega ocenimo, da je velikost nehomogenosti polja v našem magnetu približno

\begin{equation*}
    \langle \delta B \rangle = (30 \pm 5)\,\,\mathrm{\mu T},
\end{equation*}

Drugi efekt pa je obračanje posameznih jedrskih magnetnih momentov nazaj proti osi zunanjega $B$ polja. Obračanje karakterizira čas $T_1$, sicer kot

\begin{equation}
    M_z = M_0 \left( 1 - e^{-t/T_1} \right).
    \label{eq:T1}
\end{equation}

Da ga izmerimo, sistem vzbudimo z dvema sunkoma $\pi/2$. Prvi sunek nastavi $M_z$ vseh jeder na $0$. V času $\tau$ do drugega sunka se del magnetizacije relaksira nazaj v smer zunanjega polja. Ta del magnetizacije obrne drugi sunek nazaj v ravnino $xy$. Preostanek magnetizacije, ki pa se ni relaksiral, temveč je precesiral v ravnini $xy$, pa obrne naprej, še enkrat za $\pi/2$, v celoti za $\pi$. Amplituda precesije, ki jo izmerimo po drugem sunku (za čas $\tau$ po prvem), je torej sorazmerna delu magnetizacije, ki se je relaksirala (raste od $0$ proti $M_0$) v času $\tau$.

Meritve za vodo z ioni in navadno vodo vidimo na sliki~\ref{fig:T1} zgoraj in spodaj. Na meritve prilagodimo funkcijo oblike~(\ref{eq:T1}), pri čemer $\tau$ igra vlogo časa, ki je bil na voljo za relaksacijo. S prilagajanjem izračunamo relaksacijske čase

\begin{align*}
    T_1(\text{voda z ioni}) = (3 \pm 1)\,\mathrm{ms}, \\
    T_1(\text{voda}) = (0.61 \pm 0.06)\,\mathrm{ms}.
\end{align*}

\begin{figure}
    \begin{center}
    \includegraphics{T1-ions}
    \includegraphics{T1-noions}
    \end{center}
    \caption{Signal po drugem $\pi/2$ sunku v odvisnosti od časovnega zamika med sunkoma $\tau$. Časovni konstanti $T_1$ za vodo z ioni in navadno vodo določimo s prilagajanjem funkcije oblike~(\ref{eq:T1}).}
    \label{fig:T1}
\end{figure}

\subsection{Zaporedna \texorpdfstring{$\pi/2$ in $\pi$}{pi/2 in pi} sunka ter spinski odmev}

Omenili smo mehanizem s časovno konstanto $T_2^*$, po katerem $xy$ zaradi nehomogenosti $\delta B$ razpade še preden se lahko relaksira nazaj v smer zunanjega polja. A tudi če je polje absolutno homogeno, se zaradi efektov nižjega reda precesije magnetnih momentov desinhronizirajo. Ta bolj osnovna desinhronizacija poteka s časovno konstanto $T_2$.

\begin{figure}[ht]
    \begin{center}
        \includegraphics[width=0.55\textwidth]{diagram}
    \end{center}
    \caption{Shematski prikaz $\pi$ obratov jedrskega momenta.}
    \label{fig:diagram}
\end{figure}

Da izmerimo ta efekt, se moramo znebiti efekta $\delta B$. V ta namen bi lahko takoj po sunku $\pi/2$ obrnili predznak nehomogenosti. To bi v vrtečem sistemu jedra~(glej sliko~\ref{fig:diagram}) rotaciji s frekvenco

\begin{equation*}
    \omega_i = \gamma\,\delta B,
\end{equation*}

spremenilo smer in ga vrnilo v začetno orientacijo~(spet slika~\ref{fig:diagram}). A lažje kot z obratom polja enako dosežemo z $\pi$ obratom samega momenta, kot je to prikazano na sliki~\ref{fig:diagram}. Če smo omenjeni sunek $\pi$ priskbeli $\tau$ po prvem sunku $\pi/2$, se bo obrnjen moment v začetno lego (v katero ga je spravil prvi sunek $\pi/2$) vrnil ravno po še enem dodatnem času $\tau$. Signalu, ki ga zaznamo ob tej \blockquote{vrnitvi} pravimo spinski odmev.

Podobno kot pri meritvi $T_1$ je tu $\tau$ čas, ki je do drugega sunka na voljo za razpad začetnega $\pi/2$ stanja. Amplituda precesije po drugem sunku pa je sorazmerna $xy$ magnetizaciji, ki se je relaksirala (od $M_0$ do $0$) v času $\tau$.

\begin{figure}[H]
    \begin{center}
    \includegraphics{T2-ions}
    \end{center}
    \caption{Logaritmirane meritve signala proste precesije, ki je sorazmeren magnetizaciji v $xy$ ravnini. Naklon premice je $T_2$, časovna konstanta razpada magnetizacije v $xy$ ravnini zaradi statističnih efektov.}
    \label{fig:T2-ions}
\end{figure}

Meritve za vodo z ioni vidimo na sliki~\ref{fig:T2-ions}. Ker je relaksacija eksponenten razpad, lahko meritve preprosto logaritmiramo in s prilagojeno premico izračunamo časovno konstanto

\begin{equation*}
    T_2 = (1.1 \pm 0.1).
\end{equation*}

\end{document}
