% !TEX program = xelatex

% Base
\documentclass{article}
\usepackage[a4paper,margin=1in]{geometry}

% Locale
\usepackage{polyglossia}
\setdefaultlanguage[localalph=true]{slovenian}
\usepackage[autostyle]{csquotes}
\DeclareQuoteAlias{german}{slovene}

% Bibliography
\usepackage[backend=biber,style=numeric]{biblatex}
\addbibresource{/home/martin/literatura.bib}

% Math
\usepackage{amsmath}
\usepackage{amssymb}
\usepackage{siunitx}
\usepackage{physics}

% Imported pdf_tex figures
\usepackage{graphicx,import}
\usepackage{subfig}
\usepackage{color}

% Hyperlinks
\usepackage{hyperref}
\usepackage[svgnames]{xcolor}

% Pgfplots
\usepackage{amssymb}

% Styling
\numberwithin{equation}{section}
\setlength{\skip\footins}{1.5cm}

% Differential
\newcommand{\diff}{\mathrm{d}}

% "Defined as" symbol
\usepackage{mathtools}
\newcommand{\das}{\vcentcolon=}
\newcommand{\asd}{=\vcentcolon}

\setcounter{page}{0}
\setcounter{section}{8}
\setcounter{subsection}{0}

\title{Spektrometrija žarkov $\gamma$ s scintilacijskim spektrometrom}
\author{Martin Šifrar}

\begin{document}

\maketitle

\subsection{Naloga}

\begin{enumerate}
    \item Prilagodite valovod na generator mikrovalov.
    \item Izmerite frekvenco valovanja s pomočjo v valovod vgrajenega resonatorja.
    \item Posnemite rodove klistronovega delovanja v odvisnosti od odbojne napetosti.
    \item Izmerite moči, ki jih porablja termistor v vrhovih najmočnejših rodov.
    \item Z osciloskopom posnemite krivulji ubranosti za valovod, ki je zaključen z bremenom, in za kratko sklenjeni valovod.
\end{enumerate}

\subsection{Meritve}

\begin{figure}
    \begin{center}
        \includegraphics[width=0.5\textwidth]{setting-1.jpg}
    \end{center}
    \caption{Nastavitev ubiralke, pri kateri je signal maksimalen. To nastavitev, ki jo obdržimo skozi vse meritve.}
    \label{fig:setting}
\end{figure}

Na valovod pritrdimo trobljo in nastavimo ubiralko in odbojno napetost tako, da je signal maksimalen. To nastavitev obdržimo skozi vse kasnejše meritve.
Frekvenco valovanja določimo tako, da z vrtenjem mikrometerskega vijaka spreminjamo dimenzijo resonančne votline in opazujemo signal na merilni sondi. Ko signal pade (pravzaprav opazimo dva minimuma, ki sta zelo blizu skupaj), je vijak v položaju

\begin{table}
    \centering
    \begin{tabular}{c|r}
        lega & $\nu\,[\mathrm{GHz}]$ \\
        \hline
        100 & 10.0 \\
        300 & 9.0 \\
        500 & 8.0
    \end{tabular}
    \caption{Umeritvena tabela mikrometerskega vijaka resonančne votline.}
    \label{tab:calib}
\end{table}

\begin{equation*}
    x_\mathrm{vijak} = 410 \pm 1,
\end{equation*}

iz česar preko umeritvene tabele~\ref{tab:calib} izračunamo frekvenco

\begin{equation*}
    \nu = (11.05 \pm 0.01)\,\mathrm{GHz}.
\end{equation*}

Zdaj prečešemo celoten obseg odbojne napetosti in poiščemo vse rede delovanja klistrona. Nekateri izmed teh so šibkejši in jih težje opazimo. Pri teh odbojnih napetostih nato z bolometrom izmerimo še moč $P_m$, tj. moč, ki se porablja na termistorju. Meritve so predstavljene v prvih dveh stolpci tabele~\ref{tab:s-meas}. Ko kasneje izračunamo ubranost $s$, bomo pravo moč $P$~(zadnji stolpec tabele~\ref{tab:s-meas}) izračunali kot

\begin{equation}
    P = \frac{P_m}{1 - |r_R|^2},
    \label{eq:P}
\end{equation}

pri čemer je rekflekcijski koeficient $r_R$ definiran kot

\begin{equation*}
    |r_R|^2 = \left( \frac{1-s}{1+s} \right)^2.
\end{equation*}

Z valovodom, zaključenim z bolometrom, pomerimo umeritveno krivuljo. To naredimo tako, da počasi premaknemo voziček s sondo, zraven pa na osciloskopu beležimo položaj vozička in izmerjeno vrednost na sondi (jakost). Enako meritev ponovimo še z valovodom, ki je zaključem s steno (je kratko staknjen). Dobljena poteka sta vidna na sliki~\ref{fig:traces}. Da iz meritev potenciometra lahko izračunamo razdaljo, izmerimo še razliko med ekstremoma položaja vozička

\begin{equation*}
    l = (4.6 \pm 0.2)\,\mathrm{cm}.
\end{equation*}

\begin{figure}
    \begin{center}
    \includegraphics{traces.pdf}
    \end{center}
    \caption{Poteki jakosti na sondi in potenciometra (položaja sonde) za valovod, zaključen z bolometrom (zgoraj) in valovod, zaključen z steno (spodaj).}
    \label{fig:traces}
\end{figure}

\begin{figure}
    \begin{center}
    \includegraphics{s.pdf}
    \end{center}
    \caption{Krivulji ubranosti za valovod, zaključen z bolometrom (zelena) in valovod, zaključen z steno (oranžna). S črno črtkano črto so narisane parabole, ki jih prilagodimo posameznim vrhovom, da natančneje odčitamo njihov položaj.}
    \label{fig:s}
\end{figure}

Prvo iz korena razmerja amplitud v krivulji ubranosti (slika~\ref{fig:s}) izračunamo parameter ubranosti $s$, ki ima vrednost

\begin{equation*}
    s = 0.494 \pm 0.001.
\end{equation*}

Zdaj z odčitano razdaljamo $x_\mathrm{min}$ in valovno dolžino $\lambda$ izračunamo produkt

\begin{equation*}
    \beta x_\mathrm{min} = 2\pi \frac{x_\mathrm{min}'}{\lambda'},
\end{equation*}

katerega vrednost je

\begin{equation*}
    \beta x_\mathrm{min} = 1.10 \pm 0.08.
\end{equation*}

Končno izračunamo še relativno reaktanco bremena kot

\begin{equation*}
    \frac{\eta_R}{Z_0} = \frac{ (s^2 - 1) \tan \beta x_\mathrm{min} }{ 1 + s^2 \tan^2 \beta x_\mathrm{min} }.
\end{equation*}

Dobimo vrednost

\begin{equation*}
    \frac{\eta_R}{Z_0} = -0.76 \pm 0.02,
\end{equation*}

in njegovo relativno rezistanco kot $\frac{\xi_R}{Z_0} = \left( 1 - \frac{\eta_R}{Z_0} \tan \beta  x_\mathrm{min} \right) s$. Dobimo

\begin{equation*}
    \frac{\xi_R}{Z_0} = 1.23 \pm 0.04. 
\end{equation*}

Absolutna vrednost relativne (proti $Z_0$) impedance bremena je torej

\begin{equation*}
    \left| \frac{Z_R}{Z_0} \right| = 1.45 \pm 0.04.
\end{equation*}

\begin{table}
    \centering
    \begin{tabular}{c|r|r}
        $U_o\,[\mathrm{V}]$ & $P_m\,[\mathrm{mW}]$ & $P\,[\mathrm{mW}]$ \\
        \hline
        -30.2 & 0.10 & 0.11 \\
        -55.7 & 0.22 & 0.25 \\
        -92.8 & 0.38 & 0.43 \\
        \textbf{-148.3} & \textbf{0.56} & \textbf{0.63} \\
        -230.1 & 0.52 & 0.59 \\
        -359.8 & 0.38 & 0.43
    \end{tabular}
    \caption{Odbojne napetosti v vseh najdenih redovih delovanja klistrona in moč na termistorju $P_m$. V zadnjem stolpcu je predstavljena tudi prava moč valovanja $P$, ki jo izračunamo s pomočjo ubranosti $s$ preko enačbe~\ref{eq:P}.}
    \label{tab:s-meas}
\end{table}

\end{document}
