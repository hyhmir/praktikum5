% !TEX program = xelatex

% Base
\documentclass{article}
\usepackage[a4paper,margin=1in]{geometry}

% Locale
\usepackage{polyglossia}
\setdefaultlanguage[localalph=true]{slovenian}
\usepackage[autostyle]{csquotes}
\DeclareQuoteAlias{german}{slovene}

% Bibliography
\usepackage[backend=biber,style=numeric]{biblatex}
\addbibresource{/home/martin/literatura.bib}

% Math
\usepackage{amsmath}
\usepackage{amssymb}
\usepackage{siunitx}

% Imported pdf_tex figures
\usepackage{graphicx,import}
\usepackage{subfig}
\usepackage{color}

% Hyperlinks
\usepackage{hyperref}
\usepackage[svgnames]{xcolor}

% Pgfplots
\usepackage{amssymb}

% Styling
\numberwithin{equation}{section}
\setlength{\skip\footins}{1.5cm}

% Differential
\newcommand{\diff}{\mathrm{d}}

% "Defined as" symbol
\usepackage{mathtools}
\newcommand{\das}{\vcentcolon=}
\newcommand{\asd}{=\vcentcolon}

\title{Hallov pojav}

\author{Martin Šifrar}

\begin{document}

\maketitle

\section{Uvod}

\begin{figure}
    \begin{center}
        \includegraphics[width=0.35\textwidth]{band-structure.png}
    \end{center}
    \caption{Valenčni in prevodni pas v n-dopiranem polprevodniku.}
    \label{fig:band-structure}
\end{figure}

\begin{figure}
    \begin{center}
        \includegraphics[width=0.6\textwidth]{limits.png}
    \end{center}
    \caption{Diagram, kjer jasno vidimo visoko in nizko-temperaturno limito izrazov~(\ref{eq:5},~\ref{eq:6},~\ref{eq:7}).}
    \label{fig:limits}
\end{figure}

V čistem polprevodniku je gostota prevodnih elektronov enaka

\begin{equation}
    n_p(T) = \frac{1}{4} \left( \frac{2m_e kT}{\pi h^2} \right)^{3/2} \exp \left\{ -\frac{E_g}{2kT} \right\},
    \label{eq:5}
\end{equation}

pri čemer je $E_g$ energijska reža med valenčnim in prevodnim pasom. V n-dopiranem polprevodniku zveza v nizko-temperaturni limiti postane

\begin{equation}
    n_p(T) = N_d \frac{1}{4} \left( \frac{2m_e kT}{\pi h^2} \right)^{3/2} \exp \left\{ -\frac{E_d}{2kT} \right\},
    \label{eq:6}
\end{equation}

pri čemer je $N_d$ gostota donorskih primesi, $E_d$ pa reža od prevodnega pasa navzdol proti nivoju donorskih elektronov~(glej sliko~\ref{fig:band-structure}). Če je termična energija dovolj velika ($kT > E_d$), prevladajo v prevodnem pasu donorski elektroni, ki so kar v celoti \blockquote{sublimirani} v prevodni pas. Tedaj velja, da je gostota prevodnih elektronov preprosto

\begin{equation}
    n_p(T) = N_d.
    \label{eq:7}
\end{equation}

\section{Naloga}

\begin{enumerate}
    \item Izmeri temperaturno odvisnost Hallove napetosti vzorca polprevodnika tipa n na temperaturnem območju med $20\,\mathrm{^\circ C}$ in $80\,\mathrm{^\circ C}$.
    \item Nariši graf Ohmske upornosti $R$ v odvisnosti od temperature $T$.
    \item Nariši graf Hallove konstante $R_H$ v odvisnosti od temperature T .
    \item S pomočjo enačbe (4) nariši graf $\ln(n_p)$ v odvisnosti od $1/kT$ .
    \item Določi vrsto nosilcev naboja v germanijevem vzorcu na tem temperaturnem območju. Preveri ustreznost enačb~(\ref{eq:5},~\ref{eq:6},~\ref{eq:7}).
\end{enumerate}

\section{Meritve}

Vzorec napaja baterija z napetostjo

\begin{equation*}
    U_0 = (0.89 \pm 0.01)\,\mathrm{V}.
\end{equation*}

Pri različnih temperaturah pomerimo napetost in tok. Oboje izmerimo v obeh orientacijah magnetnega polja, napetosti v prvi in drugi orientaciji označimo $U_1$ in $U_2$. Tok se iz ene v drugo orientacijo znotraj napake ne spremeni. Za obe orientaciji izračunamo Ohmsko upornost vzorca~(glej~sliko~\ref{fig:both}).

\begin{figure}[h]
    \begin{center}
    \includegraphics{both.pdf}
    \end{center}
    \caption{Ohmske upornosti, izračunane v eni in drugi orientaciji vzorca. Kar nas bo kasneje zanimalo, je razlika napetosti $(U_1 - U_2)/2$, ki predstavlja Hallovo napetost. Ker je temperaturni potek toka znotraj napake identičen za prvo in drugo orientacijo, torej imamo le en tok $I = I_1 = I_2$, je razlika med krivuljama sorazmerna kasneje predstavljeni $R_H$.}
    \label{fig:both}
\end{figure}

Hallovo napetost lahko izračunamo iz napetosti v eni in drugi orientaciji kot

\begin{equation*}
    U_H = \frac{U_1 - U_2}{2},
\end{equation*}

s čimer se znebimo Ohmskega padca napetosti, ki ni odvisen od orientacije magnetnega polja (oz. vzorca v polju). Iz dobljene napetosti~(slika~\ref{fig:U-by-T}~levo~zgoraj) izračunamo Hallovo konstanto, definirano kot

\begin{equation}
    R_H = \frac{cU_H}{IB},
    \label{eq:R_H}
\end{equation}

pri čemer je $c = 0.95\,\mathrm{mm}$ debelina našega vzorca, $B$ pa gostota magnetnega polja $B = 0.173\,\mathrm{T}$. Temperaturno odvisnost $R_H(T)$ vidimo na sliki~\ref{fig:U-by-T} spodaj.

\begin{figure}
    \begin{center}
    \includegraphics{U-by-T.pdf}
    \end{center}
    \caption{Potek Hallove napetosti $U_H = (U_1 - U_2)/2$ in toka (zgoraj). Spodaj je izračunana Hallova konstanta, ki je neke vrste \blockquote{prečni upor} vzorca. Predznak konstante je odvisen od tega, kakšno konvencijo izberemo. Pomembno pa je pomniti, da se pri spremembi predznaka nosilcev naboja obrne tudi njen predznak.}
    \label{fig:U-by-T}
\end{figure}

Poleg načina, na katerega smo Hallovo konstanto definirali v~(\ref{eq:R_H}), iz samih lastnosti sistema sledi, da velja

\begin{equation*}
    R_H = -\frac{1}{ne_0},
\end{equation*}

pri čemer je $n$ gostota nosilcev naboja. Tako lahko izračunamo gostoto nosilcev naboja in jo kot v simboličnem diagramu~\ref{fig:limits} predstavimo v odvisnosti od $1/kT$ na sliki~\ref{fig:lnn-by-1-by-kT}.

\begin{figure}
    \begin{center}
    \includegraphics{lnn-by-1-by-kT.pdf}
    \end{center}
    \caption{Logaritmirane}
    \label{fig:lnn-by-1-by-kT}
\end{figure}

V visoko-temperaturni limiti, torej za manjše vrednosti $1/kT$, je logaritem $n$ po $1/kT$ premica z naklonom $-E_g/2$. To sledi direktno iz enačbe~\ref{eq:5}, ki predstavlja visoko-temperaturno limito celotnega sistema. Iz naših meritev izračunamo, da je naklon premice

\begin{equation*}
    m = (0.44 \pm 0.03)\,\mathrm{ev},
\end{equation*}

iz česar dobimo energijsko režo

\begin{equation*}
    E_g = (0.88 \pm 0.07)\,\mathrm{eV}.
\end{equation*}

Za naklon premice v nizko-temperaturni limiti bi morali vzorec ohladiti znatno pod sobno temperaturo, torej enačbe~\ref{eq:7} z našimi meritvami ne moremo preveriti. Vidimo pa, da se~\ref{fig:lnn-by-1-by-kT} na desni, pri nižjih temperaturah, nekoliko izravna.

\end{document}
