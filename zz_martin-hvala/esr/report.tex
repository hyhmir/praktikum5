% !TEX program = xelatex

% Base
\documentclass{article}
\usepackage[a4paper,margin=1in]{geometry}

% Locale
\usepackage{polyglossia}
\setdefaultlanguage[localalph=true]{slovenian}
\usepackage[autostyle]{csquotes}
\DeclareQuoteAlias{german}{slovene}

% Bibliography
\usepackage[backend=biber,style=numeric]{biblatex}
\addbibresource{/home/martin/literatura.bib}

% Math
\usepackage{amsmath}
\usepackage{amssymb}
\usepackage{siunitx}

% Imported pdf_tex figures
\usepackage{graphicx,import}
\usepackage{subfig}
\usepackage{color}

% Hyperlinks
\usepackage{hyperref}
\usepackage[svgnames]{xcolor}

% Pgfplots
\usepackage{amssymb}

% Styling
\numberwithin{equation}{section}
\setlength{\skip\footins}{1.5cm}

% Differential
\newcommand{\diff}{\mathrm{d}}

% "Defined as" symbol
\usepackage{mathtools}
\newcommand{\das}{\vcentcolon=}
\newcommand{\asd}{=\vcentcolon}

\title{Elektronska spinska resonanca}
\author{Martin Šifrar}

\begin{document}

\maketitle

\section{Naloga}

\begin{enumerate}
    \item Z vzorcem DPPH kot merjencem določi g-faktor prostega elektrona in razmerje $B_0/\nu$.
    \item Izmeri širino absorpcijske črte.
\end{enumerate}

\section{Meritve}

Z osciloskopom neposredno izmerimo frekvenco regenerirajočega oscilatorja

\begin{equation*}
    \nu_0 = (69.4 \pm 0.2)\,\mathrm{Hz}.
\end{equation*}

Naprej uporabimo fazni detektor, ki z množenjem in odstranitvijo visoke nosilne frekvence meri napetost, sorazmerno odvodu absorpcijskega spektra. Po različnih magnetnih gostotah $B_0$ (merimo tok skozi večjo tuljavo) pomerimo odvod spektra. Meritve so prikazane na sliki~\ref{fig:spectra} zgoraj.

\begin{figure}
    \begin{center}
        \includegraphics{spectra.pdf}
    \end{center}
    \caption{Odvisnost $\Delta U(B_0)$ (zgoraj) in njen integral (spodaj), ki je sorazmeren z absorpcijo valovanja v vzorcu. Povezane črte predstavljajo funkcije, prilagojene na meritve odvoda spektra. Na spodnji sliki je prikazan integral meritev, dodatno premaknjen za konstanto, ki se najbolje ujema z integralom funkcije, prilagojene na meritve v zgornji sliki.}
    \label{fig:spectra}
\end{figure}

Zanima nas povprečje in širina absorpcijske črte, ki jo definiramo kar kot razdaljo med ekstremoma odvoda\footnote{Za odvod Gaussovke je to kar $2\sigma$.} Na meritve odvoda prilagodimo odvod Gaussovke z dvema prostima parametroma

\begin{equation*}
    f(x) \propto (x - \mu) e^{-(x - \mu)^2 / 2\sigma^2},
\end{equation*}

z dodatnim pogojem enotske $L^2$ norme. Kot vidimo na sliki~\ref{fig:spectra} zgoraj, so repi izmerjenega spektra predebeli za tako obliko, katere repi padajo z $e^{-x^2}$. Zato prilagodimo še odvod funkcije, katere repi padajo z $e^{-x}$. Ena taka funkcija je

\begin{equation*}
    g(x) \propto \frac{\sinh (x/a)}{\cosh^2 (x/a)},
\end{equation*}

spet z enotsko $L^2$ normo. Kot vidimo na sliki~\ref{fig:spectra} in v tabeli~\ref{tab:d-I}, ta funkcijska oblika nekoliko bolje razloži naše meritve, kot to vidimo v deležu nerazložene variance.

\begin{table}
    \centering
    \begin{tabular}{r|r|rr|rr}
        $\nu\,[\mathrm{MHz}]$ & $d_\mathrm{minmax}\,[\mathrm{mV}]$ & $d_\mathrm{Gauss}\,[\mathrm{mV}]$ & nerazložena $\mathrm{Var}$ & $d_\mathrm{sinh}\,[\mathrm{mV}]$ & nerazložena $\mathrm{Var}$ \\
        \hline
        $79.2 \pm 0.2$ &  9.0 & $11.3 \pm 0.2$ & 2.5\% & $10.0 \pm 0.2$ & 1.1\% \\
        $85.0 \pm 0.2$ & 12.0 & $13.2 \pm 0.2$ & 1.5\% & $11.7 \pm 0.1$ & 0.8\% \\
        $90.2 \pm 0.2$ & 12.0 & $13.2 \pm 0.2$ & 1.6\% & $11.8 \pm 0.2$ & 0.8\% \\
    \end{tabular}
    \caption{}
    \label{tab:d-I}
\end{table}

\begin{table}
    \centering
    \begin{tabular}{r|rr}
        $\nu\,[\mathrm{MHz}]$ & $B_0[\mathrm{mT}]$ & $d[\mathrm{mT}]$ \\
        \hline
        $79.2 \pm 0.2$ & $2.9 \pm 0.1$ & $0.107 \pm 0.005$ \\
        $85.0 \pm 0.2$ & $3.1 \pm 0.1$ & $0.126 \pm 0.005$ \\
        $90.2 \pm 0.2$ & $3.3 \pm 0.1$ & $0.126 \pm 0.005$ \\
    \end{tabular}
    \caption{}
    \label{tab:d-B}
\end{table}

Dobljene sredinske vrednosti in širine absorpcijske črte v enotah toka pretvorimo v magnetno gostoto. Pri tem uporabimo diagonalo tuljave, kar nam doprinese precej veliko napako.

Z dobljenimi vrednostmi $B_0$ za tri različne frekvence (tabela~\ref{tab:d-B}), lahko narišemo sliko~\ref{fig:by-freq} in skozi točke prilagodimo premico, katere naklon je $k = B_0/\nu$, njegov inverz pa

\begin{equation*}
    \frac{\nu}{B_0} = (26 \pm 3)\,\mathrm{GHz/T}.
\end{equation*}

Inverz naklona $\nu/B_0$ je iz osnovne lastnosti sistema kar $g\eta_B/h$, iz česar izračunamo giromagnetno razmerje

\begin{equation*}
    g = (1.8 \pm 0.2)\,\mathrm{GHz/T}
\end{equation*}

\begin{figure}
    \begin{center}
        \includegraphics{by-freq.pdf}
    \end{center}
    \caption{Odvisnost gostote $B_0$ z največjo absorpcijo v odvisnosti od frekvence.}
    \label{fig:by-freq}
\end{figure}

\end{document}
